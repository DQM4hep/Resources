%%%%%%%%%%%%%%%%%%%%%%% file template.tex %%%%%%%%%%%%%%%%%%%%%%%%%
%
% This is a template file for Web of Conferences Journal
%
% Copy it to a new file with a new name and use it as the basis
% for your article
%
%%%%%%%%%%%%%%%%%%%%%%%%%% EDP Science %%%%%%%%%%%%%%%%%%%%%%%%%%%%
%
%%%\documentclass[option]{webofc}
%%% "twocolumn" for typesetting an article in two columns format (default one column)
%
\documentclass{webofc}
\usepackage[varg]{txfonts}   % Web of Conferences font
%
% Put here some packages required or/and some personnal commands
%
%
\begin{document}
%
\title{DQM4hep}
%
% subtitle is optionnal
%
\subtitle{A generic data quality monitoring framework for HEP}

\author{
\firstname{R\'emi} \lastname{Ete}\inst{1}\fnsep\thanks{\email{remi.ete@desy.de}}
\and
\firstname{Antoine} \lastname{Pingault}\inst{2}\fnsep\thanks{\email{antoine.pingault@ugent.be}}
}

\institute{
DESY, Notkestra\ss e 85, 22607 Hamburg, Germany
\and
Ghent University, Department of Physics and Astronomy Proeftuinstraat 86, B-9000 Gent, Belgium
}

\abstract{%
Data quality monitoring is the first step to the certification of the recorded data for off-line physics analysis.
Dedicated monitoring framework have been developed by many experiments in the past and usually rely on the event data
model (EDM) of the experiment, leading to a strong dependency on the data format and storage. We present here a generic
data quality monitoring system, DQM4HEP, that has been developed without any assumption on the EDM. This increases the
code maintenance, the portability across different experiments and re-usability for future experiment.
We present the framework architecture and the various tools provided by the software package as well as various
performances such as memory usage, stability and network bandwidth. We give an overview of the different experiments
using DQM4HEP and the foreseen integration in future other experiments. We finally present the ongoing and future
software development for DQM4HEP and long term prospects.
}
%
\maketitle
%
\section{Introduction}
\label{sec:intro}

Online monitoring systems are crucial while taking experimental data. The main tasks of these systems are to provide a quick overview of the detector and sub-detector status, and to evaluate the quality of these data. More technically, the online framwork should provide the following ingredients:

\begin{itemize}
  \item a DAQ link to access the data and talk to the run control,
  \item a dedicated online analysis framework,
  \item a visualization system to show the resulting analysis products
\end{itemize}

Such systems are generally (but not always) decoupled from the data quality monitoring done offline on recorded data or simulated data. By looking at the different data quality monitoring software available on the HEP market, we see that for most of them the architecture rely on the event data model (EDM) of the underlying experiment. This can be seen as a limitation, as the software becomes non-reusable by other experiments.

In this paper, we present a new generic framework for online and offline monitoring called DQM4hep. After introducing the framework architecture and specifications, the different components of DQM4hep are described. The software still being in development, we conclude with a few perspective and incoming features.

\section{The DQM4hep framework}
\label{sec:framework}

\subsection{Software architecture and specifications}
\label{subsec:arch}



\begin{itemize}
  \item Event data model abstraction
  \item Distributed system
  \item Low cost DAQ link (no DAQ perturbation)
  \item Speed (the more events processed, the better)
  \item Distributed system for the online part
  \item hardware requirement: network bandwidth, memory capacity (N histos in memory)
  \item Memory stability: over hours/days
  \item Something else ?
\end{itemize}

\subsection{The core components}
\label{subsec:core}

\begin{itemize}
  \item The plugin system
  \item The event data model abstraction
  \item The event streamer
  \item The monitor element
  \item The quality test + report
  \item Others ?
\end{itemize}

\subsection{The online system}
\label{subsec:online}

\begin{itemize}
  \item The analysis module
  \item The standalone module
  \item The collectors: events and monitor elements
  \item Interface to DAQ systems: run control and event source
  \item Others ?
\end{itemize}

\subsection{The online visualization}
\label{subsec:vis}

\begin{itemize}
  \item The job control Gui
  \item The Qt Gui
  \item Toward a web interface (work in progress)
\end{itemize}

+ screenshots

\section{Detectors using DQM4hep}
\label{sec:detectors}

\begin{itemize}
  \item SDHCAL test-beams
  \item Combined test beam with SiWECal
  \item AHCal test-beams
  \item DREAM test-beams ???
\end{itemize}

\section{Conclusion}
\label{sec:conclusion}

Conclusion \\
+ extensions:

\begin{itemize}
  \item Web interface
  \item Continuous integration
\end{itemize}

\section{Aknowledgment}
\label{sec:aknowledgment}

If DREAM calorimeter, AIDA2020 aknowledgments


% \section{Introduction}
% \label{intro}
% Your text comes here. Separate text sections with
% \section{Section title}
% \label{sec-1}
% For bibliography use \cite{RefJ}
% \subsection{Subsection title}
% \label{sec-2}
% Don't forget to give each section, subsection, subsubsection, and
% paragraph a unique label (see Sect.~\ref{sec-1}).
%
% For one-column wide figures use syntax of figure~\ref{fig-1}
% \begin{figure}[h]
% % Use the relevant command for your figure-insertion program
% % to insert the figure file.
% \centering
% %\includegraphics[width=1cm,clip]{tiger}
% \caption{Please write your figure caption here}
% \label{fig-1}       % Give a unique label
% \end{figure}
%
% For two-column wide figures use syntax of figure~\ref{fig-2}
% \begin{figure*}
% \centering
% % Use the relevant command for your figure-insertion program
% % to insert the figure file. See example above.
% % If not, use
% \vspace*{5cm}       % Give the correct figure height in cm
% \caption{Please write your figure caption here}
% \label{fig-2}       % Give a unique label
% \end{figure*}
%
% For figure with sidecaption legend use syntax of figure
% \begin{figure}
% % Use the relevant command for your figure-insertion program
% % to insert the figure file.
% \centering
% \sidecaption
% %\includegraphics[width=5cm,clip]{tiger}
% \caption{Please write your figure caption here}
% \label{fig-3}       % Give a unique label
% \end{figure}
%
% For tables use syntax in table~\ref{tab-1}.
% \begin{table}
% \centering
% \caption{Please write your table caption here}
% \label{tab-1}       % Give a unique label
% % For LaTeX tables you can use
% \begin{tabular}{lll}
% \hline
% first & second & third  \\\hline
% number & number & number \\
% number & number & number \\\hline
% \end{tabular}
% % Or use
% \vspace*{5cm}  % with the correct table height
% \end{table}
%
% BibTeX or Biber users please use (the style is already called in the class, ensure that the "woc.bst" style is in your local directory)
% \bibliography{name or your bibliography database}
%
% Non-BibTeX users please use
%
\begin{thebibliography}{}
%
% and use \bibitem to create references.
%
\bibitem{RefJ}
% Format for Journal Reference
Journal Author, Journal \textbf{Volume}, page numbers (year)
% Format for books
\bibitem{RefB}
Book Author, \textit{Book title} (Publisher, place, year) page numbers
% etc
\end{thebibliography}

\end{document}

% end of file template.tex

%<div id='footer'><table width='100%'><tr><td class='right'><a href='http://fusioninventory.org/'><span class='copyright'>FusionInventory 9.1+1.0 | copyleft <img src='/glpi/plugins/fusioninventory/pics/copyleft.png'/>  2010-2016 by FusionInventory Team</span></a></td></tr></table></div>
